\documentclass[a4paper]{article}
\setlength{\columnsep}{40pt}
\usepackage{graphicx} 
\usepackage[a4paper,margin=0.5in]{geometry}
\usepackage{amsmath}
\usepackage{booktabs}
\usepackage{hyperref}

\title{Bobcats, Coyotes \& Foxes}
\author{Salvin Chowdhury}
\date{\today}

\begin{document}

\setlength{\intextsep}{0pt} 
\setlength{\textfloatsep}{5pt} 

\maketitle
\tableofcontents
\newpage

\section{Introduction}
This paper is for a midterm project dedicated towards replicating the results from a paper on a morphometric modeling approach to distinguishing among
bobcat, coyote, and gray fox scats. This midterm project requires determining the best approach for addressing research questions as shown in Section 1.2.

Understanding the distribution and behavior of various species is fundamental to ecology, as it helps ecologists understand the relationship between
organsims and their environments. There is a lot ecological signifigance behind tracking such species, as researchers can see precisely when individual 
animals depart from one location and arrive at another. Such insights reveal a individual animal's seasonal movements, feeding locations and more. Such
information is valuable because the more ecologists know about animals' seasonal usage of habitats, the more they can protect the areas the animals need to
survive.

Given that adult Gray Foxes have a greater average body length than adult Bobcats, and a smaller average body length than adult Coyotes, we hypothesize 
that Gray Fox scat will exhibit a significantly larger average diameter than Bobcat scat, but a smaller average diameter than Coyote scat.

\subsection{Background Information}
Conservation exologists monitor species populations to assess the health of the ecosystem and just how effective conservation strategies are. While there are
several methods for estimating population sizes, such as mark-recapture and camera-trap surveys, an alternative approach to this is analyzing biological 
remnants such as scat, as performed by Dr. Reid.

In a 2015 study by Dr. Rachel Reid, Reid had investigated whether morphologica, biogeochemical and contextual traits distinguish between bobcat, coyote and
fox scat. Morphological traits are the most cost-effective for field identification, while biogeochemical traits require laboratory analysis. 

\subsection{The Research Questions}
The purpose of this paper is to analyze the data to evaluate whether specific morphological or biogeochemical traits can reliably differentiate species. We 
also research biological and ecological factors that explain the observed patterns. The research questions are listed as follows:
\begin{itemize}
    \item Which (if any) morphological and biogeochemical traits distinguish between originating species of the scat samples?
    \item Why do you think those traits differ across species?
\end{itemize}

\subsection{The Diet, Habitat \& Physical Characteristics of Bobcats}
Bobcats are mostly carnivoruous. Their diet consists of a variety of animals, such as rabbits, rodents, wood rats birds, insects and many more. They also
will consume plant material such as grass. Bobcats will also hunt pets or small livestock such as chicken if they're not kept in a secure enclosure. Bobcats
can be found in diverse habitats throughout California. Suitable Bobcat habitat includes vegetation types, brushy stages of low and mid-elevation conifer, 
forests and desert environments. Interestingly, Bobcats prefer areas with dense bush cover. When it comes to physical characteristics, Bobcats are 
medium-sized cats with muscular bodies. They weight between 12 and 25 pounds. Bobcats have a round face with ruffs of fur on the side of the head. They have
pointed ears, and appear to be approximately one quarter of the size of a mountain lion.

\subsection{The Diet, Habitat \& Physical Characteristics of Coyotes}
The diet of a coyote consists mainly only mice, rats, ground squirrels, gophers, rabbits and carrion. They also eat insects and birds. In rural areas of
California, they prey heavily on sheep, cattle and poultry. In urban and suburban areas, domestic cats, dogs and other pets can be food items. Coyotes can
be found in the hotter drier regions of California. They can also be found in mountainous or humid areas in the state. Some physical characteristics of the 
coyote are that they are medium sized animals that belong to the Dog family. They can weigh between 22 to 25 pounds on average, with males being the larger
sex. They also have large erect ears, slender muzzle and a bushy tail. Lastly, they have a distinctive voice and are proficient predators.

\subsection{The Diet, Habitat \& Physical Characteristics of Gray Foxes}
The diet of a gray fox consists mainly of small rodents, birds and berries. They will also eat insects, eggs, acorns and fungi. Gray foxes can be found in
populating coastal or mountain forests at lower elevations. They rarely dig their own dens, but will instead rest in crevices, under boulders and in hollow
logs. Some physical characteristics of Gray Foxes are that it has short legs, a silvery-gray coat with patches of yellow, brown, rust, or white on the 
throat and the belly. Black tipped guard hairs form a dark line down its back to the tip of the tail.

\newpage

\subsection{Population Estimation Methods for Conservation}
Next, we look at the different population estimation methods for conservation. We also look at their strengths and weaknesses. 

\subsubsection{Mark \& Recapture Benefits and Drawbacks}
For mobile organsisms, we use a method called Mark \& Recapture. This method involves marking a sample of captured animals and then releasing them back into 
the environment to allow them to mix with the rest of the population. A common issue with mark-recapture methods is that the process of capturing and marking
the animals changes their behavior. This is otherwise known as a trap response.

\subsubsection{Qudrants Benefits and Drawbacks}
A quadrant is a frame used in ecology to isolate a standard unit of area for the study of the distribution of an item over a large area. Quadrats typically 
occupy an area of 0.25 $m^2$, and are traditionally square. Some benefits it is quick, inexpensive and portable. Some of the disadvantages are it is not 
very accurate and the sample can seem unintentionally biased.

\subsubsection{Transects Benefits and Drawbacks}
A transect is a straight line that cuts through a natural landscape so that standardized observations and measurements can be made. Some advantages of using
transects are it is quick, inexpensive and portable. A disadvantage of transects is that it is often used in inapproriate situations, so care must be taken 
when deciding whether or not to use a transect.



\newpage

\begin{thebibliography}{99}

    \bibitem{bobcat} California Department of Fish and Wildlife. \textit{Bobcat}. Available at: \url{https://wildlife.ca.gov/Conservation/Mammals/Bobcat#574393461-bobcat-biology}. Accessed: 2025-03-17.
    \bibitem{coyote_diet} U.S. Forest Service. \textit{The Diet of the Coyote}. Available at: \url{https://www.fs.usda.gov/detailfull/mendocino/learning/nature-science/?cid=FSBDEV3_004458#:~:text=The%20diet%20of%20the%20coyote,sheep%2C%20cattle%2C%20and%20poultry}. Accessed: 2025-03-17.
    \bibitem{gray_fox} California Living Museum. \textit{Gray Fox}. Available at: \url{https://calmzoo.org/blog/animals/gray-fox/}. Accessed: 2025-03-17.
    \bibitem{population_ecology} LibreTexts. \textit{Population Ecology Research Methods}. Available at: \url{https://bio.libretexts.org/Courses/Gettysburg_College/01%3A_Ecology_for_All/09%3A_The_Ecology_of_Populations/9.02%3A_Population_Ecology_Research_Methods}. Accessed: 2025-03-17.
    \bibitem{quadrat_transects_pitfall} Uyir. \textit{The Advantages and Disadvantages of Quadrats, Transects, and Pitfall Traps}. Available at: \url{https://www.uyir.at/explore-learn/eshaaan/the-advantages-and-disadvantages-of-quadrats-transects-and-pitfall-traps}. Accessed: 2025-03-17.
    \bibitem{tracking_wildlife} National Aquarium. \textit{Connecting the Dots: The Science of Tracking Wildlife}. Available at: \url{https://aqua.org/stories/2022-12-05-connecting-dots-the-science-of-tracking-wildlife}. Accessed: 2025-03-17.


    
\end{thebibliography}

\end{document}

